
\documentclass[11pt, aspectratio=169]{beamer}
\usefonttheme{professionalfonts}% font de LaTeX
\usetheme{default}
\usetheme{Ilmenau} 
\usepackage[utf8]{inputenc}
\usepackage{latexsym} 
\usepackage{amsmath}
\usepackage{amssymb}
\usepackage{graphicx}
\usepackage{enumerate}
%\usepackage[default]{gfsneohellenic}



\setbeamertemplate{navigation symbols}{}
\logo{\includegraphics[width=1cm,height=1cm,keepaspectratio]{creativecommons.jpg}}


\begin{document}

%%%%%%%%%%%%%%%%%%%%%%%%%%%%%%%%%%%%%

\begin{frame}

\frametitle{\huge\centerline{1.- Trazados básicos}}

\begin{enumerate}

\item Dados dos puntos, construye el segmento que los une y traza su mediatriz.

\item Dada una recta y un punto contenido en dicha recta, traza la perpendicular a la recta por dicho punto.

\item Dada una recta y un punto exterior, traza la perpendicular a la recta que pasa por dicho punto.

\item Dada una recta y un punto exterior, traza la paralela a la recta que pasa por dicho punto.

\item Dados dos puntos, traza el segmento que los une y dibuja la perpendicular al segmento que pasa por uno de sus extremos.

\item Dadas dos semirrectas, traza la bisectriz del ángulo que forman.

\end{enumerate}

\end{frame}

%%%%%%%%%%%%%%%%%%%%%%%%%%%%%%%%%%%%

\begin{frame}

\frametitle{\huge\centerline{2.- Rectas notables de un triángulo}}

\begin{enumerate}

\item Dado un triángulo, traza sus tres mediatrices y comprueba que se cortan en un punto denominado \textbf{circuncentro}. Construye  la circunferencia circunscrita al triángulo.

\item Dado un triángulo, traza sus tres bisectrices y comprueba que se cortan en un punto denominado \textbf{incentro}. Traza la circunferencia inscrita.

\item Traza las tres medianas de un triángulo y comprueba que se cortan en un punto denominado \textbf{baricentro}.

\item Construye las tres alturas de un triángulo y comprueba que se cortan en un punto denominador \textbf{ortocentro}.

\end{enumerate}

\end{frame}

%%%%%%%%%%%%%%%%%%%%%%%%%%%%%%%%%%%%



\begin{frame}

\frametitle{\huge\centerline{3.- Polígonos regulares 1}}

\begin{enumerate}

\item Dado un segmento  construye un triángulo equilátero que tenga por lado dicho segmento. 

\item Dibuja un segmento y construye un cuadrado cuyo lado sea dicho segmento. 

\item Dibuja un segmento y construye un cuadrado cuya  diagonal sea dicho segmento.

\item Dibuja un segmento y  construye un hexágono regular cuyo lado sea dicho segmento. 

\item Dibuja un octógono regular a partir de uno de sus lados. 

\end{enumerate}

\end{frame}

%%%%%%%%%%%%%%%%%%%%%%%%%%%%%%%%%%
\begin{frame}

\frametitle{\huge\centerline{4.- Polígonos regulares 2}}

\begin{enumerate}

\item Dada una circunferencia, inscribir un triángulo equilátero.

\item Dada una circunferencia, inscribir un cuadrado.

\item Dada una  circunferencia, inscribir un hexágono regular.

\item Dada una  circunferencia, inscribir un octógono regular.

\end{enumerate}

\end{frame}

%%%%%%%%%%%%%%%%%%%%%%%%%%%%%%%%%%


\begin{frame}

\frametitle{\huge\centerline{5.-  Circunferencias}}

\begin{enumerate}

\item Dado el diametro construir la circunferencia.

\item Construir con regla y compás la circunferencia que pasa por tres puntos no alineados.

\item Dado un punto de un circunferencia, traza la tangente en dicho punto.

\item Dada una circunferencia y un punto exterior $P$, traza una circunferencia de centro $P$ y que sea tangente a la circunferencia dada.

\item Dado un punto exterior a la circunferencia, traza las dos tangentes a la circunferencia que pasen por dicho punto.

\end{enumerate}

\end{frame}

%%%%%%%%%%%%%%%%%%%%%%%%%%%%%%%%%%%%%%%%%

\begin{frame}

\frametitle{\huge\centerline{6.-  Construcción de triángulos}}

\begin{enumerate}



\item Dibujar un triángulo conociendo los tres lados.

\item Dibujar un triángulo equilátero conociendo la altura.

\item Triángulo rectángulo conociendo los dos catetos.

\item Triángulo rectángulo conociendo un cateto y la hipotenusa.

\item Triángulo isósceles conociendo el lado desigual y los lados iguales.

\end{enumerate}

\end{frame}


%%%%%%%%%%%%%%%%%%%%%%%%%%%%%%%%%%%%%%%%%

\begin{frame}

\frametitle{\huge\centerline{7.-  Cónicas}}

\begin{enumerate}

\item Construye una elipse y comprueba que la suma de los radiovectores es constante.

\item Construye una hipérbola y comprueba que la resta de los radiovectores es constante.

\item Construye una parábola y comprueba que las distancias al foco y a la directriz son iguales.

\item Construir una parábola como un lugar geométrico.

\item Construir una elipse como un lugar geométrico.


\end{enumerate}

\end{frame}



%%%%%%%%%%%%%%%%%%%%%%%%%%%%%%%%%%%%


\begin{frame}

\frametitle{\huge\centerline{8.-  Vectores}}



\begin{enumerate}

\item Dados los puntos $A=(2,1)$ y $B=(5,3)$ calcula el vector que une $A$ con $B$, de manera algebraica y de manera geométrica.

\item Dados los vectores $u=(2,1)$ y $v=(-3,4)$ calcula su suma y su resta.

\item Calcular el módulo y el ángulo que forma con el eje $x$ el vector $v=(3,5)$.

\item Calcular el ángulo que forman los vectores $u=(2,4)$ y $v=(-3,2)$.

\item Calcula el producto escalar de los vectores del ejercicio anterior.

\item Multiplica el vector $u=(2,1)$ por un número arbitrario y comprueba que tiene la misma dirección, pero que el sentido cambia cuando el número es negativo.

\end{enumerate}

\end{frame}



%%%%%%%%%%%%%%%%%%%%%%%%%%%%%%%%%%%%%%%%%

\begin{frame}

\frametitle{\huge\centerline{9.-  Geometría analítica I}}

\begin{enumerate}

\item Dados los puntos $A=(3,5)$ y $B=(-2,3)$ calcula gráfica y algebraicamente el punto medio.

\item Dado el punto $A=(3,1)$ halla, geométrica y analíticamente, el simétrico de $A$ respecto del punto $P=(2,4)$.

\item Halla la ecuación de la recta que pasa por el punto $A=(3,2)$ y tiene como vector director $v=(2,-3)$.

\item Halla la ecuación de la recta que pasa por los puntos $A=(3,-1)$ y $B=(-2,4)$.

\item Dado el punto $A=(3,2)$ y la recta $y=2x-1$, calcula el punto simétrico respecto a la recta.

\end{enumerate}

\end{frame}

%%%%%%%%%%%%%%%%%%%%%%%%%%%%%%%%%%%%%%%%%


\begin{frame}

\frametitle{\huge\centerline{10.-  Geometría analítica II}}



\begin{enumerate}

\item Halla el punto de corte de  $8x-2y=20$ y $3x+2y=13$.

\item Dada la recta $y=3x-2$ y el punto $A=(2,-1)$ calcula la recta paralela que pasa por dicho punto. Comprueba que las pendientes coinciden.

\item La recta $3x-4y=2$ junto con los ejes coordenados forma un triángulo. Calcula el área de dicho triángulo.

\item Dado el triángulo de vértices $A=(2,3)$, $B=(-1,2)$ y $C=(4,-2)$ calcula el baricentro por el método geométrico y por el algebraico.

\item Dado el triángulo de vértices $A=(2,3)$, $B=(-1,2)$ y $C=(4,-2)$ calcular los puntos medios de dos de sus lados y formar la recta que los une. Comprobar que dicha recta es paralela al otro lado. 

\end{enumerate}

\end{frame}

%%%%%%%%%%%%%%%%%%%%%%%%%%%%%%%%%%%



%%%%%%%%%%%%%%%%%%%%%%%%%%%%%%%%%%%%%%%%%


\begin{frame}

\frametitle{\huge\centerline{11.-  Circunferencia y geometría analítica}}



\begin{enumerate}

\item Encuentra la ecuación de la circunferencia de radio $r=3$ y que tiene el centro en el punto $C=(1,3)$.

\item Encuentra la ecuación de la circunferencia con centro $C=(1,3)$ y que pasa por el punto $P=(3,4)$.

\item Dada la circunferencia  $x^2+y^2-6x-4y-3=0$, encontrar su centro y su radio.

\item Encontrar la ecuación de la circunferencia que tiene centro en el punto $C=(1,3)$ y que es tangente a la recta $2x-3x=1$.

\item Encuentra, de modo aproximado, el valor de $b$ para que la recta $y=x+b$ sea tangente a la circunferencia $x^2+y^2=4$. 

\end{enumerate}

\end{frame}

%%%%%%%%%%%%%%%%%%%%%%%%%%%%%%%%%%%%%%%%%


\begin{frame}

\frametitle{\huge\centerline{12.-  Cónicas y geometría analítica}}



\begin{enumerate}

\item Encuentra los semiejes, la distancia focal y la excentricidad de  

\vspace{-0.3cm}
\[
\frac{x^2}{16}+\frac{y^2}{4}=1
\]

\vspace{-0.2cm}

\item Analiza en la ecuación $\frac{x^2}{a^2}+\frac{y^2}{b^2}=1$, el significado geométrico de $a$ y $b$.


\item Una elipse tiene focos en los puntos $F=(-2,0)$ y $F'=(2,0)$ y su semieje mayor mide 3. Encuentra la ecuación de la elipse.

\item Una elipse tiene los focos en los puntos $F=(-2,0)$ y $F'=(2,0)$ y pasa por el punto $P=(4,2)$. Halla la ecuación de la elipse.

\item Calcula la hipérbola que tiene los mismos focos que la elipse anterior y que pasa por el mismo punto.  Resuelve el sistema.

\end{enumerate}

\end{frame}

%%%%%%%%%%%%%%%%%%%%%%%%%%%%%%%%%%%%%%%%%


\begin{frame}

\frametitle{\huge\centerline{13.-  Números complejos}}



\begin{enumerate}

\item Dados los números complejos $z=2+i$ y $w=-3+i$ calcula su suma, su resta y su producto.

\item Dados los números complejos $z=2_{45^0}$ y $w=3_{60^0}$ multiplícalos. Comprueba que el argumento del producto es la suma de los argumentos.

\item Dado el número complejo $z=2+3i$ calcula su raíz cuadrada y su raíz cúbica.

\item Calcula potencias enteras de la unidad imaginaria y observa su significado geométrico.

\item Dado un número complejo arbitrario, multiplícalo por la unidad imaginaria y analiza el significado geométrico.

\end{enumerate}

\end{frame}


%%%%%%%%%%%%%%%%%%%%%%%%%%%%%%%%%%%%%%%%%


\begin{frame}

\frametitle{\huge\centerline{14.-  Representación de funciones}}



\begin{enumerate}

\item Dibuja la gráfica de la función $y=x^3-2x^2+2$.

\item Dibuja la gráfica de la función $y=mx+n$ y analiza el significado geométrico de $m$ y de $n$.

\item Dibuja la gráfica de la función $y=a(x-b)^2+c$ y analiza el significado geométrico de los parámetros.

\item Dibuja una función definida a trozos.
\[
f(x)=\begin{cases}
x-3 & \text{si }x<-1\\
x^2 & \text{si } -1<x<2\\
5 & \text{si } x>2
\end{cases}
\]

\item Calcula la suma de dos ondas senoidales con desfase y comprueba que se pueden anular.

\end{enumerate}

\end{frame}




%%%%%%%%%%%%%%%%%%%%%%%%%%%%%%%%


\begin{frame}

\frametitle{\huge\centerline{15.-  Derivadas}}



\begin{enumerate}

\item Calcula la derivada del polinomio $f(x)=x^3 - 3x^2 + x + 3$. Calcula la segunda derivada.

\item Calcula la recta tangente a la curva anterior en un punto arbitrario.

\item Comprueba, de modo empírico, que la recta tangente en los máximos y en los mínimos es horizontal.

\item Encuentra las soluciones de la derivada y verifica si en dichos puntos la función posee extremos o carece de ellos.

\item Construye la derivada de la función como un lugar geométrico.


\end{enumerate}

\end{frame}


%%%%%%%%%%%%%%%%%%%%%%%%%%%%%%%%%%%%%%%%%


\begin{frame}

\frametitle{\huge\centerline{16.-  Integrales}}



\begin{enumerate}

\item Calcula la integral indefinida de las función  $x\cdot cos(x)$.

\item Calcula la integral de la función anteriores en el intervalo $[-3,2]$.

\item Calcula la integral de una función entre dos puntos arbitrarios.

\item Calcula el área de la función comprendida entre dos puntos de corte consecutivos de la función anterior y la función $x^2-2x-1$.

\item Calcula la suma inferior de una función y compárala con el valor de la integral.

\end{enumerate}

\end{frame}

%%%%%%%%%%%%%%%%%%%%%%%%%%%%%%%%%%%%%%%%%


\begin{frame}

\frametitle{\huge\centerline{17.-  Estudio de funciones}}

\begin{enumerate}

\item Dibuja la gráfica de la función $y=x^3 - 3x^2 + x + 3$ y analiza sus características principales.

\item Calcula los máximos, los mínimos y los puntos de inflexión de la función anterior utilizando comandos de Geogebra.

\item Calcula los máximos, los mínimos y los puntos de inflexión de la función anterior utilizando derivadas.

\item Analiza la curvatura de la función anterior.

\item Encuentra, y dibuja las asíntotas de la función:
\vspace{-0.3cm}
\[
\frac{2x^2-3x+2}{x^2-1}
\]

\end{enumerate}

\end{frame}


%%%%%%%%%%%%%%%%%%%%%%%%%%%%%%%%%%%%%%%%%


\begin{frame}

\frametitle{\huge\centerline{18.-  Transformaciones I}}



\begin{enumerate}

\item Dibuja un triángulo y realiza una simetría respecto a un punto. Comprueba que las distancias y los ángulos se conservan.

\item Haz lo mismo que en el ejercicio anterior pero reflejando el triángulo en una  recta, comprobando que se conservan las mismas cantidades.

\item Rota un triángulo y comprueba las cantidades conservadas.

\item Traslada un triángulo con un vector.

\item Realiza una homotecia de razón $k$ y verifica que los ángulos se conservan pero las longitudes se multiplican por $k$ y las áreas por $k^2$.

\end{enumerate}

\end{frame}


%%%%%%%%%%%%%%%%%%%%%%%%%%%%%%%%%%%%%%%%%


\begin{frame}

\frametitle{\huge\centerline{19.-  Transformaciones II}}



\begin{enumerate}

\item Dado un triángulo y su trasladado, calcula el vector de traslación.

\item Dado un triángulo y su reflejado en un punto, calcula el centro de reflexión.

\item Dado un triángulo y su reflejado en una recta, encuentra dicha recta.

\item Dado un triángulo y su rotado, encuentra el centro de rotación y el ángulo de rotación.

\item Dado un triángulo y su homotético, calcula el centro de la homotecia y su razón.


\end{enumerate}

\end{frame}


%%%%%%%%%%%%%%%%%%%%%%%%%%%%%%%%%%%%%%%%%





\begin{frame}

\frametitle{\huge\centerline{20.-  Ángulos y circunferencias}}



\begin{enumerate}



\item Relaciona el ángulo central de una circunferencia con el ángulo inscrito (aquel que tiene el vértice en la circunferencia).

\item Comprobar que en el caso del ángulo central llano, el ángulo inscrito es recto.

\item Utilizar lo anterior para construir triángulos rectángulos.

\item Traza una cuerda de una circunferencia. Cualquier ángulo inscrito cuyos vértices sean los extremos de dicha cuerda. Traza un diámetro que pase por uno de los extremos de la cuerda y relaciona los ángulos.

\item Construir el arco capaz de un  ángulo $\alpha$.


\end{enumerate}

\end{frame}




%%%%%%%%%%%%%%%%%%%%%%%%%%%%%%%%%%%%%%%%%


\begin{frame}

\frametitle{\huge\centerline{21.-  Potencia y eje radical}}



\begin{enumerate}

\item Calcula por  métodos distintos la potencia de un punto respecto a una circunferencia.

\item Calcula, geométrica y algebraicamente, el eje radical de dos circunferencias: la primera de centro $A=(2,1)$ y radio 3 y la segunda de centro $B=(7,2)$ y radio 4.

\item Calcula el eje radical de dos circunferencias exteriores.

\item Calcula el eje radical de dos circunferencias tangentes exteriores.

\item Calcula el eje radical de dos circunferencias interiores (no concéntricas).


\end{enumerate}

\end{frame}


%%%%%%%%%%%%%%%%%%%%%%%%%%%%%%%%%%%%%%%%%


\begin{frame}

\frametitle{\huge\centerline{22.-  Inversión respecto a una circunferencia I}}



\begin{enumerate}

\item Comprueba que al multiplicar las distancias al centro de dos puntos inversos se obtiene siempre el radio al cuadrado. Comprueba que los tres puntos están alineados.  Comprueba que en un punto y su inverso están siempre en distinto ``lado'' de la circunferencia. Si el punto está en la circunferencia, coincide con su inverso. Comprobar que el inverso del inverso es el punto de partida.

\item Construye de manera geométrica el inverso de un punto  situado dentro de la circunferencia. Hacer lo mismo con un punto exterior.

\item Sean $A$ y $A'$ dos puntos inversos respecto de $c$. Cualquier circunferencia $c'$ que pase por los puntos $A$ y $A'$ es ortogonal a $c$.

\item Sean $A, A'$ y $B, B'$ dos puntos simétricos respecto a una circunferencia. \linebreak 
Entonces los cuatro puntos están situados en una misma circunferencia.










\end{enumerate}

\end{frame}

%%%%%%%%%%%%%%%%%%%%%%%%%%%%%%%%%%%%%%%%%


\begin{frame}

\frametitle{\huge\centerline{23.-  Inversión respecto a una circunferencia II}}



\begin{enumerate}

\item Comprueba que la inversa de cualquier recta que no pase por el centro de inversión es una circunferencia que pasa por el centro de inversión y comprueba el recíproco. Estudiar el caso de recta secante y tangente.

\item La inversión transforma circunferencias en circunferencias.

\item Comprueba que la inversión conserva ángulos.

\item Comprobar que las rectas que pasan por el centro y las circunferencias que pasan por dos puntos simétricos son objetos invariantes.


\end{enumerate}

\end{frame}


\end{document}

%%%%%%%%%%%%%%%%%%%%%%%%%%%%%%%%%%%%%%%%%





\begin{frame}

\frametitle{\huge\centerline{Circunferencia y recta polar}}

\begin{enumerate}

\item La polar como recta que une los puntos de tangencia.

\item La polar como la recta que une los puntos de corte de la circunferencia con la circunferencia de diámetro OP

\item La polar trazando dos secantes.

\item La polar trazando dos secantes simétricas.

\item La polar utilizando la fórmula de la circunferencia

\item La polar utilizando el punto inverso.

\item La polar como lugar geométrico de circunferencias que cortan ortogonalmente

\end{enumerate}

\end{frame}


%%%%%%%%%%%%%%%%%%%%%%%%%%%%%%%%%%%%%%%%%


\begin{frame}

\frametitle{\huge\centerline{Razón aurea}}

\begin{enumerate}

\item Aparición en un triángulo isósceles con un ángulo de 36 grados.

\item Construir un rectángulo aurea a partir de un cuadrado.

\end{enumerate}

\end{frame}





Matrices y geogebra.

Matriz de determinante 1 no cambia el  area.

Matriz con otro determinante cambia el area.

Matrices de rotación.

Encontrar geométricamente los autovectores de una matriz.











Dibuja un cuadrilátero arbitrario. Calcula los puntos medios de los lados y forma un nuevo cuadrilátero con dichos puntos. Comprueba que es un paralelogramo.

Comprobar teorema de Pitágoras

Comprobar teorema de Napoleón

Comprobar teorema de Viviani

comprobar teorema de la circunferencia de 9 puntos.

Construir las raíces de los números naturales con regla y compás


Construir una perpendicular por un punto exterior aplicando el arco capaz

División de un ángulo recto en tres partes iguales.

Aproximación a $\pi$ por polígonos regulares.

?????? Catenaria  

Necesidad mecánica del teorema de Pitágoras.

El tamaño de papel A0.

Construir Fibonacci con cuadrados.

Construir la espiral logarítmica.




Resuelve geométricamente la ecuación $x^3-4x-1=0$.

Resuelve geométricamente la ecuación $x^3-6x+7=\cos(x)$.

Resuelve geométricamente el sistema (dos ecuaciones no lineales)

Construye geométricamente el número $\sqrt{2}$.

Construye geométricamente el número de oro.

Dados los polinomios $p(x)=x^3-2x+3$ y $q(x)=3x^2-5x^2-1$, calcula su suma y su producto, desarrollando los resultados.

Desarrolla el cuadrado del binomio $(x^2-5x)^2$.

Factoriza el polinomio $x^3-7x^2+8x+16$.



%%%%%%%%%%%%  Cossincal.com


\begin{frame}

\frametitle{\huge\centerline{Resolución de triángulos (http://cossincalc.com/)}}

\begin{enumerate}

\item Resolver el triángulo rectángulo de catetos 3 y 4.

\item Resolver el triángulo rectángulo de hipotenusa 25 y cateto 7.

\item Resolver el triángulo de lados 7, 9 y 11.

\item Intentar resolver el triángulo de  lados 80, 20 y 25.

\end{enumerate}



\end{frame}



\end{document}